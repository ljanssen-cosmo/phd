\documentclass[12pt]{article}

\usepackage{amsmath}
\usepackage{graphicx}
\usepackage[margin=1in]{geometry}
\usepackage{enumitem}
\usepackage{hyperref}

% for checkmark and xmark
\usepackage{pifont}
\usepackage{xcolor}   
\newcommand{\cmark}{\textcolor{green}{\ding{51}}}
\newcommand{\xmark}{\textcolor{red}{\ding{55}}} 

\setlength{\parindent}{0pt}
\setlength{\parskip}{1em}
\setlist[enumerate]{before=\vspace{-\parskip}}
\setlist[itemize]{before=\vspace{-\parskip}}

\title{Notes}
\author{Lena Janssen}
\date{}

\begin{document}

\maketitle

\subsection*{September 1, 2025}
First day: Take care of administrative and organizational tasks. The plan for this week is to read some papers and then discuss the project in more detail next week.

Started reading \textit{Particle Physics and Inflationary Cosmology} by Andrei Linde (\href{https://arxiv.org/abs/hep-th/0503203}{arXiv:hep-th/0503203}). So far, the cosmology basics do not need to be reviewed, but the QFT part should. But I will finish the first chapter of the lecture notes first. 

Helpful books for the basics:
\begin{itemize}
    \item \textit{Modern Cosmology}, Dodelson \& Schmidt \textbf{OR} \textit{Cosmology}, Baumann
    \item \textit{Quantum Field Theory}, Peskin \& Schroeder
    \item \textit{The Early Universe}, Kolb \& Turner
    \item A GR book (Carrol or Wald)
\end{itemize}

Questions:
\begin{enumerate}
    \item With what papers should I start? \cmark\\ 
    \textbf{Answer:} The lecture notes from Daan.
\end{enumerate}

\subsection*{September 2, 2025}
I will start reading the lecture notes that Daan has provided. We will have a short meeting tomorrow discussing more of the details of the project. He also mentioned that I need to get in contact with the postdoc, who is also part of the consortium. She will start in November — she doesn't know too much about the societal contribution, but it will be nice to do this together.

Open points:
\begin{enumerate}
    \item Understand the path integral formalism in QFT in order to understand the correlation function of a random field. \xmark
    \item Review bispectrum and trispectrum and how they are derived. \xmark
    \item Review spherical harmonics and the derivation of the angular correlation function of the CMB. \xmark
    \item Maybe check again the definitions of geodesics and eq. 57 in the lecture notes. \xmark
    \item SVT decomposition \xmark
\end{enumerate}

\subsection*{September 3, 2025}
Continue reading the lecture notes from Daan. Try and finish these and reread all the important sections. For now, a basic understanding is enough. No need to go into many details. The questions from yesterday still need to be worked on, but I think it makes sense to do that once I know what I need to pay close attention to. 

Another important part will be the intro essay (10-15 pages) that needs to be written within the first 6 months and will be discussed at the first R\&O conversation with Ema and Daan. At 9 months is the go/no-go interview from which is determined if I can continue the PhD. It is said that it is in most cases a "go". But make sure that to ask about this interview way in advance so that I could prepare if necessary. 

Read most of the lecture notes by Daan, but need a refresh on a couple things. It was very helpful to also read the Baumann lecture notes that also cover these topics. Still, I think I need some time reading up on the basics before I can start actually working on something. Tomorrow is the meeting with Daan, so hopefully we can come up with a plan of maybe some papers to read, so that I can prepare what I need to know in order to understand these papers. 

What I find really interesting:
\begin{enumerate}
    \item Theoretical models of inflation and how they can produce a larger signal that can potentially be observed today.
    \item Primordial gravitational waves
    \begin{itemize}
        \item Imprint that pgw leave on the CMB → special type of polarization pattern → B-mode pattern (cannot be created by scalar fluctuations)
    \end{itemize}
    \item Primordial non-Guassianities (do not understand the details yet)
    \begin{itemize}
        \item Models of inflation that predict the same power spectrum but different levels and shapes on nG
        \item three-point correlation function (not possible to calculate this with KFT)
        \item For chapter 8 in Baumann, I really need to review QFT(/QM) for the in-in formalism
    \end{itemize}
\end{enumerate}

\subsection*{September 4, 2025}
The first meeting went really well and we decided to for now start with a more "surface level" project to get started and maybe push a paper asap. Gravitational waves and the emergence of an inflationary model is maybe the first avenue. We will also have a meeting with Ema to see what she says because she is more of an expert on the theory of inflation. 

Some ideas:
Lightbird: Potential connection of the Netherlands to this project because of the production of parts. SInce lightbird measures B-modes in the CMB, this will be potentially very interesting for primordial gravitational waves. Should know about this in the next couple of months. Waiting for aüürpval of Japan. But if it works out, this could be a very new avenue that also includes some data anaylsis,w hcih would be nice
spectral -> black body of the CMB gets shifted during inflation. Both of them have worked on this before and there could be a connection. Potentiall great connection would be with Eiichiro

Collaboration connections:
Utrecht for large scale structures and maybe interacting dark energy. Also, try and find out what people are doing at the kick-off event. This will be an amazing opportunity to get to know people in the Netherlands and sort of have an "in" with people. Daan also mentioned that it could be very beneficial to have two publictaions in different areas. This could bve very helpful for the future in academia. Also, if possible, it would be nice to establish a connection to a different "pillar" in the emegrence consortium, since that is one of the reasons why it exists. 

\subsection*{September 5, 2025}
Today, I started reading the Baumann lectures on Cosmology. We also had the arxiv friday and the first group meeting. Not much happened besides that.

\subsection*{September 8, 2025}
I realized that I should brush up on my QM before I really understand quantum fluctuations during inflation. Hence I started reading Matthias QM lecture notes. I will try to read the important parts, including about symmetries. If this isn't enough, I might start reading Sakurai, or Shankhar. It would probably be helpful to also brush up on some QFT, but I will do that once I am done with this and need it.
\end{document}